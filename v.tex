% Options for packages loaded elsewhere
\PassOptionsToPackage{unicode}{hyperref}
\PassOptionsToPackage{hyphens}{url}
%
\documentclass[
]{article}
\usepackage{lmodern}
\usepackage{amssymb,amsmath}
\usepackage{ifxetex,ifluatex}
\ifnum 0\ifxetex 1\fi\ifluatex 1\fi=0 % if pdftex
  \usepackage[T1]{fontenc}
  \usepackage[utf8]{inputenc}
  \usepackage{textcomp} % provide euro and other symbols
\else % if luatex or xetex
  \usepackage{unicode-math}
  \defaultfontfeatures{Scale=MatchLowercase}
  \defaultfontfeatures[\rmfamily]{Ligatures=TeX,Scale=1}
\fi
% Use upquote if available, for straight quotes in verbatim environments
\IfFileExists{upquote.sty}{\usepackage{upquote}}{}
\IfFileExists{microtype.sty}{% use microtype if available
  \usepackage[]{microtype}
  \UseMicrotypeSet[protrusion]{basicmath} % disable protrusion for tt fonts
}{}
\makeatletter
\@ifundefined{KOMAClassName}{% if non-KOMA class
  \IfFileExists{parskip.sty}{%
    \usepackage{parskip}
  }{% else
    \setlength{\parindent}{0pt}
    \setlength{\parskip}{6pt plus 2pt minus 1pt}}
}{% if KOMA class
  \KOMAoptions{parskip=half}}
\makeatother
\usepackage{xcolor}
\IfFileExists{xurl.sty}{\usepackage{xurl}}{} % add URL line breaks if available
\IfFileExists{bookmark.sty}{\usepackage{bookmark}}{\usepackage{hyperref}}
\hypersetup{
  pdftitle={Operaciones con Matrices},
  pdfauthor={Jose A. Aranda y Esteban O. Munguía},
  hidelinks,
  pdfcreator={LaTeX via pandoc}}
\urlstyle{same} % disable monospaced font for URLs
\usepackage[margin=1in]{geometry}
\usepackage{color}
\usepackage{fancyvrb}
\newcommand{\VerbBar}{|}
\newcommand{\VERB}{\Verb[commandchars=\\\{\}]}
\DefineVerbatimEnvironment{Highlighting}{Verbatim}{commandchars=\\\{\}}
% Add ',fontsize=\small' for more characters per line
\usepackage{framed}
\definecolor{shadecolor}{RGB}{248,248,248}
\newenvironment{Shaded}{\begin{snugshade}}{\end{snugshade}}
\newcommand{\AlertTok}[1]{\textcolor[rgb]{0.94,0.16,0.16}{#1}}
\newcommand{\AnnotationTok}[1]{\textcolor[rgb]{0.56,0.35,0.01}{\textbf{\textit{#1}}}}
\newcommand{\AttributeTok}[1]{\textcolor[rgb]{0.77,0.63,0.00}{#1}}
\newcommand{\BaseNTok}[1]{\textcolor[rgb]{0.00,0.00,0.81}{#1}}
\newcommand{\BuiltInTok}[1]{#1}
\newcommand{\CharTok}[1]{\textcolor[rgb]{0.31,0.60,0.02}{#1}}
\newcommand{\CommentTok}[1]{\textcolor[rgb]{0.56,0.35,0.01}{\textit{#1}}}
\newcommand{\CommentVarTok}[1]{\textcolor[rgb]{0.56,0.35,0.01}{\textbf{\textit{#1}}}}
\newcommand{\ConstantTok}[1]{\textcolor[rgb]{0.00,0.00,0.00}{#1}}
\newcommand{\ControlFlowTok}[1]{\textcolor[rgb]{0.13,0.29,0.53}{\textbf{#1}}}
\newcommand{\DataTypeTok}[1]{\textcolor[rgb]{0.13,0.29,0.53}{#1}}
\newcommand{\DecValTok}[1]{\textcolor[rgb]{0.00,0.00,0.81}{#1}}
\newcommand{\DocumentationTok}[1]{\textcolor[rgb]{0.56,0.35,0.01}{\textbf{\textit{#1}}}}
\newcommand{\ErrorTok}[1]{\textcolor[rgb]{0.64,0.00,0.00}{\textbf{#1}}}
\newcommand{\ExtensionTok}[1]{#1}
\newcommand{\FloatTok}[1]{\textcolor[rgb]{0.00,0.00,0.81}{#1}}
\newcommand{\FunctionTok}[1]{\textcolor[rgb]{0.00,0.00,0.00}{#1}}
\newcommand{\ImportTok}[1]{#1}
\newcommand{\InformationTok}[1]{\textcolor[rgb]{0.56,0.35,0.01}{\textbf{\textit{#1}}}}
\newcommand{\KeywordTok}[1]{\textcolor[rgb]{0.13,0.29,0.53}{\textbf{#1}}}
\newcommand{\NormalTok}[1]{#1}
\newcommand{\OperatorTok}[1]{\textcolor[rgb]{0.81,0.36,0.00}{\textbf{#1}}}
\newcommand{\OtherTok}[1]{\textcolor[rgb]{0.56,0.35,0.01}{#1}}
\newcommand{\PreprocessorTok}[1]{\textcolor[rgb]{0.56,0.35,0.01}{\textit{#1}}}
\newcommand{\RegionMarkerTok}[1]{#1}
\newcommand{\SpecialCharTok}[1]{\textcolor[rgb]{0.00,0.00,0.00}{#1}}
\newcommand{\SpecialStringTok}[1]{\textcolor[rgb]{0.31,0.60,0.02}{#1}}
\newcommand{\StringTok}[1]{\textcolor[rgb]{0.31,0.60,0.02}{#1}}
\newcommand{\VariableTok}[1]{\textcolor[rgb]{0.00,0.00,0.00}{#1}}
\newcommand{\VerbatimStringTok}[1]{\textcolor[rgb]{0.31,0.60,0.02}{#1}}
\newcommand{\WarningTok}[1]{\textcolor[rgb]{0.56,0.35,0.01}{\textbf{\textit{#1}}}}
\usepackage{graphicx}
\makeatletter
\def\maxwidth{\ifdim\Gin@nat@width>\linewidth\linewidth\else\Gin@nat@width\fi}
\def\maxheight{\ifdim\Gin@nat@height>\textheight\textheight\else\Gin@nat@height\fi}
\makeatother
% Scale images if necessary, so that they will not overflow the page
% margins by default, and it is still possible to overwrite the defaults
% using explicit options in \includegraphics[width, height, ...]{}
\setkeys{Gin}{width=\maxwidth,height=\maxheight,keepaspectratio}
% Set default figure placement to htbp
\makeatletter
\def\fps@figure{htbp}
\makeatother
\setlength{\emergencystretch}{3em} % prevent overfull lines
\providecommand{\tightlist}{%
  \setlength{\itemsep}{0pt}\setlength{\parskip}{0pt}}
\setcounter{secnumdepth}{-\maxdimen} % remove section numbering

\title{Operaciones con Matrices}
\author{Jose A. Aranda y Esteban O. Munguía}
\date{}

\begin{document}
\maketitle

\begin{center}\rule{0.5\linewidth}{0.5pt}\end{center}

Para ingresar una matriz en \textbf{R}, se usa la función
\texttt{matrix()} dentro de ésta debemos indicar los valores que irán
dentro de la matriz, el número de filas \texttt{nrow=} y columnas
\texttt{ncol=} y por último, indicar si los valores se acomodarán por
filas o por columnas \texttt{byrow=}.

Hacemos un objeto \textbf{A} que es una matriz que contiene los números
del 1 al 9, de 3 filas y 3 columnas: Notar que por \emph{default} los
valores se acomodan por columna, en este caso el 1, 2 y 3 son la primera
colmuna, el 4, 5 y 6 la segunda y el 7, 8 y 9 la tercera

\begin{Shaded}
\begin{Highlighting}[]
\NormalTok{A\textless{}{-}}\KeywordTok{matrix}\NormalTok{(}\DecValTok{1}\OperatorTok{:}\DecValTok{9}\NormalTok{,}\DataTypeTok{nrow=}\DecValTok{3}\NormalTok{,}\DataTypeTok{ncol=}\DecValTok{3}\NormalTok{)}
\NormalTok{A}
\end{Highlighting}
\end{Shaded}

\begin{verbatim}
##      [,1] [,2] [,3]
## [1,]    1    4    7
## [2,]    2    5    8
## [3,]    3    6    9
\end{verbatim}

Si queremos que los valores esten acomodados por filas entonces usamos
el argumento \texttt{byrow=}, que solo puede tener 2 valores: TRUE o
FALSE, en este caso sera \texttt{byrow=TRUE}.

Notamos que ahora el 1, 2 y 3 es parte de la primera fila, el 4,5 y 6 la
segunda y el 7, 8 y 9 la tercera

\begin{Shaded}
\begin{Highlighting}[]
\NormalTok{A\textless{}{-}}\KeywordTok{matrix}\NormalTok{(}\DecValTok{1}\OperatorTok{:}\DecValTok{9}\NormalTok{,}\DataTypeTok{nrow=}\DecValTok{3}\NormalTok{,}\DataTypeTok{ncol=}\DecValTok{3}\NormalTok{,}\DataTypeTok{byrow=}\OtherTok{TRUE}\NormalTok{)}
\NormalTok{A}
\end{Highlighting}
\end{Shaded}

\begin{verbatim}
##      [,1] [,2] [,3]
## [1,]    1    2    3
## [2,]    4    5    6
## [3,]    7    8    9
\end{verbatim}

En este ejemplo estamos seguros que \textbf{A} es una matriz cuadrada
(mismo numero de filas y columnas). Sin embargo, esto lo podemos
comprobar con la funcion \texttt{dim()}; el resultado de ésta serán dos
valores, el primero indica el número de filas del objeto evaluado y el
segundo, el número de columnas.

\begin{Shaded}
\begin{Highlighting}[]
\KeywordTok{dim}\NormalTok{(A)}
\end{Highlighting}
\end{Shaded}

\begin{verbatim}
## [1] 3 3
\end{verbatim}

Podemos hacer operaciones con la matriz usando los signos
\texttt{+\ \ \ -\ \ \ *\ \ \ /} Si tenemos un objeto con un solo valor y
hacemos cualquiera de estas operaciones con la matriz se aplicara la
operacion para cada entrada de la matriz, es decir si sumamos A+x se
sumara 2 a cada entrada de la matriz:

\begin{Shaded}
\begin{Highlighting}[]
\CommentTok{\#suma}
\NormalTok{x\textless{}{-}}\DecValTok{2}
\NormalTok{A}\OperatorTok{+}\NormalTok{x}
\end{Highlighting}
\end{Shaded}

\begin{verbatim}
##      [,1] [,2] [,3]
## [1,]    3    4    5
## [2,]    6    7    8
## [3,]    9   10   11
\end{verbatim}

\begin{Shaded}
\begin{Highlighting}[]
\CommentTok{\#resta}
\NormalTok{A}\OperatorTok{{-}}\NormalTok{x}
\end{Highlighting}
\end{Shaded}

\begin{verbatim}
##      [,1] [,2] [,3]
## [1,]   -1    0    1
## [2,]    2    3    4
## [3,]    5    6    7
\end{verbatim}

\begin{Shaded}
\begin{Highlighting}[]
\CommentTok{\#multiplicación}
\NormalTok{A}\OperatorTok{*}\NormalTok{x}
\end{Highlighting}
\end{Shaded}

\begin{verbatim}
##      [,1] [,2] [,3]
## [1,]    2    4    6
## [2,]    8   10   12
## [3,]   14   16   18
\end{verbatim}

\begin{Shaded}
\begin{Highlighting}[]
\CommentTok{\#división}
\NormalTok{A}\OperatorTok{/}\NormalTok{x}
\end{Highlighting}
\end{Shaded}

\begin{verbatim}
##      [,1] [,2] [,3]
## [1,]  0.5  1.0  1.5
## [2,]  2.0  2.5  3.0
## [3,]  3.5  4.0  4.5
\end{verbatim}

Por otro lado, podemos hacer operaciones con matrices. En primer lugar
podemos sumar y restar dos matrices siempre que ambas tengan las mismas
dimensiones (mismo número de filas y columnas). Los valores de las
matrices se suman y se restan uno a uno, con su correspondiente en la
entrada de la otra matriz. En este ejemplo el valor a11 de la matriz
\textbf{A} se sumará con el elemento b11 de la matriz \textbf{B},
después, el elemento a12 se suma con el b12 y así sucesivamente. Debido
a lo anterior, el resultado es el mismo si se suma \texttt{A+B} que
\texttt{B+A}.

\begin{Shaded}
\begin{Highlighting}[]
\NormalTok{A\textless{}{-}}\KeywordTok{matrix}\NormalTok{(}\KeywordTok{c}\NormalTok{(}\DecValTok{1}\NormalTok{,}\DecValTok{2}\NormalTok{,}\DecValTok{2}\NormalTok{,}\DecValTok{1}\NormalTok{),}\DataTypeTok{ncol=}\DecValTok{2}\NormalTok{,}\DataTypeTok{byrow=}\NormalTok{T)}
\NormalTok{A}
\end{Highlighting}
\end{Shaded}

\begin{verbatim}
##      [,1] [,2]
## [1,]    1    2
## [2,]    2    1
\end{verbatim}

\begin{Shaded}
\begin{Highlighting}[]
\NormalTok{B\textless{}{-}}\KeywordTok{matrix}\NormalTok{(}\KeywordTok{c}\NormalTok{(}\DecValTok{1}\NormalTok{,}\DecValTok{2}\NormalTok{,}\DecValTok{3}\NormalTok{,}\DecValTok{4}\NormalTok{),}\DataTypeTok{nrow=}\DecValTok{2}\NormalTok{,}\DataTypeTok{byrow=}\NormalTok{T)}
\NormalTok{B}
\end{Highlighting}
\end{Shaded}

\begin{verbatim}
##      [,1] [,2]
## [1,]    1    2
## [2,]    3    4
\end{verbatim}

\begin{Shaded}
\begin{Highlighting}[]
\NormalTok{A}\OperatorTok{+}\NormalTok{B}
\end{Highlighting}
\end{Shaded}

\begin{verbatim}
##      [,1] [,2]
## [1,]    2    4
## [2,]    5    5
\end{verbatim}

Para el caso de las multiplicaciones estas solo se pueden realizar si la
primer matriz tiene el mismo número de filas que la segunda matriz. El
resulatdo de la multiplicación es una matriz que tiene el mismo numero
de filas de la primera matriz y el mismo número de columnas de la
segunda. Los valores de esta nueva matriz provienen de la suma de la
multiplicación de las filas de la primera matriz por las columnas de la
segunda. El orden para multiplicar y sumar las filas y columnas es el
siguiente:

En el ejemplo se multiplica la matriz \textbf{A} que es de 2x2 y la
matriz \textbf{C} de 2x3 (\textbf{A} mismo número de columnas que filas
en la matriz \textbf{C})
\includegraphics{http://www.portalhuarpe.com.ar/Medhime20/Nuevos\%20OA/SITIO\%20Matrices\%20y\%20Determinantes/UnidadMatematica/Navegable/Objeto1Def.MatricesyDeterminantes/Navegable/Imagenes/ejmultiplicacion.gif}

En \textbf{R} para indicar que estamos haciendo una multiplicación entre
matrices se utiliza el operador \texttt{\%*\%}. Debido a que los
resultados provienen de la suma de la multiplicación de filas y
columnas, no es lo mismo multiplicar \texttt{A*B} que \texttt{B*A}:

\begin{Shaded}
\begin{Highlighting}[]
\NormalTok{A\textless{}{-}}\KeywordTok{matrix}\NormalTok{(}\KeywordTok{c}\NormalTok{(}\DecValTok{4}\NormalTok{,}\DecValTok{6}\NormalTok{,}
           \DecValTok{{-}1}\NormalTok{,}\DecValTok{0}\NormalTok{),}\DataTypeTok{ncol=}\DecValTok{2}\NormalTok{, }\DataTypeTok{byrow=}\NormalTok{T)}

\NormalTok{B\textless{}{-}}\KeywordTok{matrix}\NormalTok{(}\KeywordTok{c}\NormalTok{(}\DecValTok{3}\NormalTok{,}\OperatorTok{{-}}\DecValTok{2}\NormalTok{,}
            \DecValTok{4}\NormalTok{,}\DecValTok{1}\NormalTok{),}\DataTypeTok{ncol=}\DecValTok{2}\NormalTok{, }\DataTypeTok{byrow=}\NormalTok{T)}
\NormalTok{A}\OperatorTok{\%*\%}\NormalTok{B}
\end{Highlighting}
\end{Shaded}

\begin{verbatim}
##      [,1] [,2]
## [1,]   36   -2
## [2,]   -3    2
\end{verbatim}

\begin{Shaded}
\begin{Highlighting}[]
\NormalTok{B}\OperatorTok{\%*\%}\NormalTok{A}
\end{Highlighting}
\end{Shaded}

\begin{verbatim}
##      [,1] [,2]
## [1,]   14   18
## [2,]   15   24
\end{verbatim}

\begin{Shaded}
\begin{Highlighting}[]
\NormalTok{C\textless{}{-}}\KeywordTok{matrix}\NormalTok{(}\KeywordTok{c}\NormalTok{(}\DecValTok{3}\NormalTok{,}\DecValTok{2}\NormalTok{,}\DecValTok{4}\NormalTok{,}
            \DecValTok{4}\NormalTok{,}\DecValTok{3}\NormalTok{,}\DecValTok{5}\NormalTok{,}
            \DecValTok{5}\NormalTok{,}\DecValTok{6}\NormalTok{,}\DecValTok{6}\NormalTok{),}\DataTypeTok{ncol=}\DecValTok{3}\NormalTok{, }\DataTypeTok{byrow=}\NormalTok{T)}

\NormalTok{D\textless{}{-}}\KeywordTok{matrix}\NormalTok{(}\KeywordTok{c}\NormalTok{(}\DecValTok{5}\NormalTok{,}\DecValTok{1}\NormalTok{,}\DecValTok{1}\NormalTok{,}
            \DecValTok{1}\NormalTok{,}\DecValTok{7}\NormalTok{,}\DecValTok{7}\NormalTok{,}
            \DecValTok{4}\NormalTok{,}\DecValTok{8}\NormalTok{,}\DecValTok{9}\NormalTok{),}\DataTypeTok{ncol=}\DecValTok{3}\NormalTok{, }\DataTypeTok{byrow=}\NormalTok{T)}
\NormalTok{C}\OperatorTok{\%*\%}\NormalTok{D}
\end{Highlighting}
\end{Shaded}

\begin{verbatim}
##      [,1] [,2] [,3]
## [1,]   33   49   53
## [2,]   43   65   70
## [3,]   55   95  101
\end{verbatim}

\begin{Shaded}
\begin{Highlighting}[]
\NormalTok{D}\OperatorTok{\%*\%}\NormalTok{C}
\end{Highlighting}
\end{Shaded}

\begin{verbatim}
##      [,1] [,2] [,3]
## [1,]   24   19   31
## [2,]   66   65   81
## [3,]   89   86  110
\end{verbatim}

Por último se pueden multiplicar matrices por vectores siempre que se
cumpla la regla que el primer elemento tenga el mismo número de columnas
que de filas el segundo. En \textbf{R} se llaman vectores a un conjunto
de valores que pueden ser o no numéricos. Un ejemplo sería el objeto
\textbf{x} que tiene una secuencia numérica. Este en \textbf{R} es un
vector, sin embargo cuando le preguntamos cuáles son sus dimensiones, el
resultado es \texttt{NULL}porque no tiene dimensiones, sólo es un
conjunto de números.

\begin{Shaded}
\begin{Highlighting}[]
\NormalTok{x\textless{}{-}}\KeywordTok{c}\NormalTok{(}\DecValTok{1}\NormalTok{,}\DecValTok{2}\NormalTok{,}\DecValTok{3}\NormalTok{,}\DecValTok{4}\NormalTok{,}\DecValTok{5}\NormalTok{)}
\NormalTok{x}
\end{Highlighting}
\end{Shaded}

\begin{verbatim}
## [1] 1 2 3 4 5
\end{verbatim}

\begin{Shaded}
\begin{Highlighting}[]
\KeywordTok{dim}\NormalTok{(x)}
\end{Highlighting}
\end{Shaded}

\begin{verbatim}
## NULL
\end{verbatim}

En notación matricial un vector es un arreglo numérico unidimensional
(que sólo es una fila o una columna). Para indicar a \textbf{R} que
queremos un vector en el sentido matricial debemos de hacer un objeto de
tipo matriz que solo tenga una fila o una columna según sea el caso:

\begin{Shaded}
\begin{Highlighting}[]
\CommentTok{\#vector de 3 filas, 1 columna}
\NormalTok{n\textless{}{-}}\KeywordTok{matrix}\NormalTok{(}\KeywordTok{c}\NormalTok{(}\DecValTok{5}\NormalTok{,}\DecValTok{1}\NormalTok{,}\DecValTok{4}\NormalTok{), }\DataTypeTok{ncol=}\DecValTok{1}\NormalTok{)}
\NormalTok{n}
\end{Highlighting}
\end{Shaded}

\begin{verbatim}
##      [,1]
## [1,]    5
## [2,]    1
## [3,]    4
\end{verbatim}

\begin{Shaded}
\begin{Highlighting}[]
\KeywordTok{dim}\NormalTok{(n)}
\end{Highlighting}
\end{Shaded}

\begin{verbatim}
## [1] 3 1
\end{verbatim}

\begin{Shaded}
\begin{Highlighting}[]
\NormalTok{A\textless{}{-}}\KeywordTok{matrix}\NormalTok{(}\KeywordTok{c}\NormalTok{(}\DecValTok{3}\NormalTok{,}\DecValTok{2}\NormalTok{,}\DecValTok{4}\NormalTok{,}
            \DecValTok{4}\NormalTok{,}\DecValTok{3}\NormalTok{,}\DecValTok{5}\NormalTok{,}
            \DecValTok{5}\NormalTok{,}\DecValTok{6}\NormalTok{,}\DecValTok{6}\NormalTok{), }\DataTypeTok{ncol=}\DecValTok{3}\NormalTok{,}\DataTypeTok{byrow=}\NormalTok{T)}

\NormalTok{A}\OperatorTok{\%*\%}\NormalTok{n}
\end{Highlighting}
\end{Shaded}

\begin{verbatim}
##      [,1]
## [1,]   33
## [2,]   43
## [3,]   55
\end{verbatim}

\begin{Shaded}
\begin{Highlighting}[]
\CommentTok{\#n\%*\%A}
\CommentTok{\#Error in n \%*\% A : argumentos no compatibles}
\end{Highlighting}
\end{Shaded}

En este último ejemplo la multiplicación de \texttt{A*n}se puede hacer
pero la operaciónde \texttt{n*A}no se puede porque las dimensiones de
este arreglo no son compatibles (el primer elemento no tiene el mismo
número de columnas que de filas el segundo).

\end{document}
